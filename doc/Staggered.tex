\documentclass{article}
\usepackage{eqalign}
\usepackage{bm}
\topmargin    -13mm
\oddsidemargin  0mm
\evensidemargin 0mm
\textheight   247mm
\textwidth    160mm
\def\deriv#1#2{\frac{\partial#1}{\partial#2}}
\def\dderiv#1#2{\frac{\partial^2#1}{\partial#2^2}}
\def\dV{\:{\rm d}V}
\title{Staggered fracture-elasticity solver}
\author{Knut Morten Okstad}

\begin{document}
\maketitle

%===============================================================================
\section{Introduction}
%===============================================================================

This document describes the weak form and associated linearized equations
for the staggered fracture-elasticity solver implemented in {\sl IFEM},
based on the papers by Borden~{\it et~al}.~\cite{Borden:2012},
and Gerasimov and De~Lorenzis~\cite{Gerasimov:2016}.

%===============================================================================
\section{Energy functionals}
%===============================================================================

The energy functional for the quasi-static brittle fracture problem,
using a phase-field to represent the crack geometry is in~\cite{Borden:2012}
given as follows (see Equation~(17) therein):
%
\begin{equation}
\label{eq:Ec}
\eqalign{
  E_c(\bm{u},c) \;=&\; \int\limits_\Omega
  \left[
    g(c)\Psi_0^+(\bm{\epsilon}(\bm{u})) + \Psi_0^-(\bm{\epsilon}(\bm{u}))
  \right]\dV \cr +&\;
  G_c\int\limits_\Omega\left(
  \frac{1}{4\ell_0}(1-c)^2 + \ell_0\bm{\nabla}c\cdot\bm{\nabla}c
  \right) \dV}
\end{equation}
%
where $c$ is the phase field describing the crack, i.e., it has the value 1.0
where the material is undamaged and equal to 0.0 for fully cracked material.
$g(c)=(1-k)c^2+k$ is the stress degradation function used to scale down the
tensile part of the strain energy density in the elasticity equation.
$k\ge0.0$ is a (small) stability parameter that can be used to improve the
conditioning of the resulting linear equation system.

In~\cite{Gerasimov:2016}, an alternative formulation is used, where $d=1-c$ is
used as the unknown phase field instead of $c$, i.e., it has the value 0.0 in
the undamaged material and 1.0 in the fully cracked material.
Moreover, a penalty term is introduced to ensure crack irreversibility,
and the length scale parameter $\ell_0$ is replaced by $\ell=2\ell_0$.
The energy functional therefore reads as follows (see also Equation~(3)
in~\cite{Gerasimov:2016}):
%
\begin{equation}
\label{eq:Ed}
\eqalign{
  E_d(\bm{u},d) \;=&\; \int\limits_\Omega
  \left[
    g(d)\Psi_0^+(\bm{\epsilon}(\bm{u})) + \Psi_0^-(\bm{\epsilon}(\bm{u}))
  \right]\dV \cr +&\;
  G_c\int\limits_\Omega\left(
  \frac{1}{2\ell}d^2 + \frac{\ell}{2}\bm{\nabla}d\cdot\bm{\nabla}d
  \right) \dV + \frac{1}{2\gamma}\int\limits_{{\rm CR}_{l-1}}(1-d)^2 \dV}
\end{equation}
%
where the stress degradation function now reads $g(d)=(1-k)(1-d)^2+k$.

The directional derivatives of $E_c$ and $E_d$ read, respectively:
%
\begin{equation}
\label{eq:Ecprime}
\eqalign{
  E_c'(\bm{u},c;\bm{v},w) \;:=&\;
  \deriv{E_c}{\bm{u}}\cdot\bm{v} \;+\; \deriv{E_c}{c}w \cr =&\;
  \int\limits_\Omega \left[
    g(c)\deriv{\Psi_0^+}{\bm{\epsilon}}(\bm{\epsilon}(\bm{u})) +
    \deriv{\Psi_0^-}{\bm{\epsilon}}(\bm{\epsilon}(\bm{u}))
  \right] : \bm{\epsilon}(\bm{v})\dV \cr +&\;
  \int\limits_\Omega g'(c)\Psi_0^+(\bm{\epsilon}(\bm{u}))w \dV \;+\;
  G_c\int\limits_\Omega\left(
    \frac{1}{2\ell_0}(c-1)w + 2\ell_0\bm{\nabla}c\cdot\bm{\nabla}w
  \right) \dV}
\end{equation}
%
with $g'(c)=2(1-k)c$, and
%
\begin{equation}
\label{eq:Edprime}
\eqalign{
  E_d'(\bm{u},d;\bm{v},w) \;:=&\;
  \deriv{E_d}{\bm{u}}\cdot\bm{v} \;+\; \deriv{E_d}{d}w \cr =&\;
  \int\limits_\Omega \left[
    g(d)\deriv{\Psi_0^+}{\bm{\epsilon}}(\bm{\epsilon}(\bm{u})) +
    \deriv{\Psi_0^-}{\bm{\epsilon}}(\bm{\epsilon}(\bm{u}))
  \right] : \bm{\epsilon}(\bm{v})\dV \cr +&\;
  \int\limits_\Omega g'(d)\Psi_0^+(\bm{\epsilon}(\bm{u}))w \dV \;+\;
  G_c\int\limits_\Omega\left(
  \frac{1}{\ell}dw + \ell\bm{\nabla}d\cdot\bm{\nabla}w
  \right) \dV + \frac{1}{\gamma}\int\limits_{{\rm CR}_{l-1}}(d-1)w \dV}
\end{equation}
%
with $g'(d)=2(1-k)(d-1)$.
The problem is then solved by seeking a solution
$(\bm{u},c)\in\bm{V}_1\times H^1(\Omega,[0,1])$, such that
%
\begin{equation}
\label{eq:Ecproblem}
  E_c'(\bm{u},c;\bm{v},w) = 0 \quad\forall\;
  (\bm{v},w)\in\bm{V}_0\times H^1(\Omega,[0,1])
\end{equation}
%
or alternatively, a solution
$(\bm{u},d)\in\bm{V}_1\times H^1(\Omega,[0,1])$, such that
%
\begin{equation}
\label{eq:Edproblem}
  E_d'(\bm{u},d;\bm{v},w) = 0 \quad\forall\;
  (\bm{v},w)\in\bm{V}_0\times H^1(\Omega,[0,1])
\end{equation}

%===============================================================================
\section{Residual and tangent for the $c$-formulation}
%===============================================================================

Equation~(\ref{eq:Ecproblem}) is rewritten into a system of coupled equations:
%
\begin{eqnarray}
  \label{eq:Q1c}
  Q_1(\bm{u},c;\bm{v}) &=& \int\limits_\Omega \left[
    g(c)\deriv{\Psi_0^+}{\bm{\epsilon}}(\bm{\epsilon}(\bm{u})) +
        \deriv{\Psi_0^-}{\bm{\epsilon}}(\bm{\epsilon}(\bm{u}))
    \right] : \bm{\epsilon}(\bm{v})\dV = 0 \\
  \label{eq:Q2c}
  Q_2(\bm{u},c;w) &=& \int\limits_\Omega \left[
    g'(c)\Psi_0^+(\bm{\epsilon}(\bm{u}))w + G_c\left(
    \frac{1}{2\ell_0}(c-1)w + 2\ell_0\bm{\nabla}c\cdot\bm{\nabla}w
    \right)\right] \dV \;=\; 0
\end{eqnarray}
%
To ensure crack irreversibility, the tensile strain energy density $\Psi_0^+$
in Equation~(\ref{eq:Q2c}) is replaced by a history field ${\cal H}(\bm{x},t)$,
with the property
%
\begin{eqnarray}
  \label{eq:initial-crack}
  {\cal H}(\bm{x},t_0) &=& \frac{G_c}{4\ell_0}\left(\frac{1}{C}-1\right)
  \left(1-\max\left\{\frac{\delta(\bm{x},\Gamma)}{\ell_0},1\right\}\right) \\
  \label{eq:history-field}
  {\cal H}(\bm{x},t_n) &=& \max\left\{{\cal H}(\bm{x},t_{n-1}),
  \Psi_0^+(\bm{\epsilon}(\bm{u}(\bm{x},t_n)))\right\}\;,\; n > 0
\end{eqnarray}
%
where $C$ is a constant and $\delta(\bm{x},\Gamma)$ denotes the shortest
distance from the point $\bm{x}$ to the initial crack $\Gamma$.
Moreover, Equation~(\ref{eq:Q2c}) is scaled by
$\frac{\ell}{G_c}=\frac{2\ell_0}{G_c}$ to obtain
%
\begin{equation}
\label{eq:Q2c-mod}
  Q_2(\bm{u},c;w) \;=\; \int\limits_\Omega \left[
    \frac{g'(c)\ell}{G_c}{\cal H}(\bm{x},t)w +
    (c-1)w + \ell^2\bm{\nabla}c\cdot\bm{\nabla}w \right] \dV \;=\; 0
\end{equation}

The Equations~(\ref{eq:Q1c}) and~(\ref{eq:Q2c-mod}) are solved in a
staggered manner, where in the former the phase field $c$ is assumed known and
in the latter the displacement field $\bm{u}$ is assumed known.
We then do a linearization of the functionals $Q_1$ and $Q_2$ about a certain
known configuration $(\bm{u}_0,c_0)$ for the unknown variables, i.e.
%
\begin{eqnarray}
  \label{eq:Q1clinearized}
  Q_1(\bm{u}_0+\Delta\bm{u},c_0;\bm{v}) &\approx&
  Q_1(\bm{u}_0,c_0;\bm{v}) \;+\; \deriv{Q_1}{\bm{u}}\cdot\Delta\bm{u} \;=\; 0 \\
  \label{eq:Q2clinearized}
  Q_2(\bm{u}_0,c_0+\Delta c;w) &\approx&
  Q_2(\bm{u}_0,c_0;\bm{v}) \;+\; \deriv{Q_2}{c}\Delta c\;=\; 0
\end{eqnarray}
%
from which we obtain the tangent operators
%
\begin{eqnarray}
  \label{eq:Q1ctangent}
  \deriv{Q_1}{\bm{u}}\cdot\Delta\bm{u} &=&
  \int\limits_\Omega \bm{\epsilon}(\Delta\bm{u}) : \left[
    g(c)\dderiv{\Psi_0^+}{\bm{\epsilon}}(\bm{\epsilon}(\bm{u})) +
        \dderiv{\Psi_0^-}{\bm{\epsilon}}(\bm{\epsilon}(\bm{u}))
    \right] : \bm{\epsilon}(\bm{v})\dV \\
  \label{eq:Q2ctangent}
  \deriv{Q_2}{c}\Delta c &=&
  \int\limits_\Omega \left[ \Delta c \left(
    \frac{g''(c)\ell}{G_c}{\cal H}(\bm{x},t) + 1 \right) w \;+\;
    \ell^2\bm{\nabla}(\Delta c)\cdot\bm{\nabla}w \right] \dV
\end{eqnarray}
%
Since $g''(c)=2(1-k)$ is a constant, the latter is linear in $\Delta c$ and $w$,
and independent of $c$, hence a linear solve is sufficient for
Equation~(\ref{eq:Q2clinearized}).
In the {\sl IFEM} implementation, we introduce the following variables
%
\begin{equation}
  s_1 = 1 + 2(1-k){\cal H}(\bm{x},t)\frac{\ell}{G_c} \quad\mbox{and}\quad
  s_2 = \ell^2 = 4\ell_0^2
\end{equation}
%
with which Equation~(\ref{eq:Q2c-mod}) can be written
%
\begin{equation}
  \label{eq:Q2c-IFEM}
  Q_2(\bm{u},c;w) \;=\; \int\limits_\Omega \left[
    (s_1c-1)w + s_2\bm{\nabla}c\cdot\bm{\nabla}w \right] \dV \;=\; 0
\end{equation}

Alternatively to using the history field, we may include a penalty term to
enforce the crack irreversibility as in Equation~(\ref{eq:Ed}), i.e.:
%
\begin{equation}
\label{eq:Ec+penaly}
\eqalign{
  E_c(\bm{u},c) \;=&\; \int\limits_\Omega
  \left[
    g(c)\Psi_0^+(\bm{\epsilon}(\bm{u})) + \Psi_0^-(\bm{\epsilon}(\bm{u}))
  \right]\dV \cr +&\;
  G_c\int\limits_\Omega\left(
  \frac{1}{4\ell_0}(1-c)^2 + \ell_0\bm{\nabla}c\cdot\bm{\nabla}c
  \right) \dV \;+\; \frac{1}{2\gamma}\int\limits_{{\rm CR}_{l-1}}c^2 \dV}
\end{equation}
%
such that the alternatives to Equations~(\ref{eq:Q2c-mod})
and~(\ref{eq:Q2ctangent}) become, respectively
%
\begin{equation}
  \label{eq:Q2c+pen}
  Q_2(\bm{u},c;w) \;=\; \int\limits_\Omega \left[
    \frac{g'(c)\ell}{G_c}\Psi_0^+(\bm{\epsilon}(\bm{u}))w +
    (c-1)w + \ell^2\bm{\nabla}c\cdot\bm{\nabla}w \right] \dV \;+\;
  \frac{1}{\gamma}\int\limits_{{\rm CR}_{l-1}}\frac{\ell}{G_c}cw \dV \;=\; 0
\end{equation}
%
and
%
\begin{equation}
  \label{eq:Q2ctangent+pen}
  \deriv{Q_2}{c}\Delta c \;=\;
  \int\limits_\Omega \left[ \Delta c \left(
    \frac{g''(c)\ell}{G_c}\Psi_0^+(\bm{\epsilon}(\bm{u})) + 1 \right) w \;+\;
    \ell^2\bm{\nabla}(\Delta c)\cdot\bm{\nabla}w \right] \dV \;+\;
  \frac{1}{\gamma}\int\limits_{{\rm CR}_{l-1}}\Delta c\frac{\ell}{G_c}w \dV
\end{equation}
%
Equation~(\ref{eq:Q2c+pen}) can be rewritten as Equation~(\ref{eq:Q2c-IFEM})
where the parameter $s_1$ now reads
%
\begin{equation}
  s_1 = 1 + 2(1-k)\Psi_0^+(\bm{\epsilon}(\bm{u}))\frac{\ell}{G_c} +
  \left\{\begin{array}{ll}
  \frac{\ell}{\gamma G_c} & \forall\,\bm{x}\in{\rm CR}_{l-1} \\[1mm]
  0 & \forall\,\bm{x}\notin{\rm CR}_{l-1}
  \end{array}\right.
\end{equation}

%===============================================================================
\section{Residual and tangent for the $d$-formulation}
%===============================================================================

Equation~(\ref{eq:Edproblem}) is rewritten into a system of coupled equations:
%
\begin{eqnarray}
  \label{eq:Q1d}
  Q_1(\bm{u},d;\bm{v})\! &=& \!\!\int\limits_\Omega \left[
    g(d)\deriv{\Psi_0^+}{\bm{\epsilon}}(\bm{\epsilon}(\bm{u})) +
        \deriv{\Psi_0^-}{\bm{\epsilon}}(\bm{\epsilon}(\bm{u}))
        \right] : \bm{\epsilon}(\bm{v})\dV \;=\; 0 \\
  \label{eq:Q2d}
  Q_2(\bm{u},d;w)\! &=& \!\!\int\limits_\Omega \left[
    g'(d)\Psi_0^+(\bm{\epsilon}(\bm{u}))w + G_c\left(
    \frac{1}{\ell}dw + \ell\bm{\nabla}d\cdot\bm{\nabla}w
    \right)\right]\!\dV + \frac{1}{\gamma}\int\limits_{{\rm CR}_{l-1}}\!\!(d-1)w \dV
  \;=\; 0
\end{eqnarray}
%
These two equations are solved in a staggered manner,
where in Equation~(\ref{eq:Q1d}) the phase field $d$ is assumed known and in
Equation~(\ref{eq:Q2d}) the displacement field $\bm{u}$ is assumed known.
We then do a linearization of the functionals $Q_1$ and $Q_2$ about a certain
known configuration $(\bm{u}_0,d_0)$ for the unknown variables, i.e.
%
\begin{eqnarray}
  \label{eq:Q1dlinearized}
  Q_1(\bm{u}_0+\Delta\bm{u},d_0;\bm{v}) &\approx&
  Q_1(\bm{u}_0,d_0;\bm{v}) \;+\; \deriv{Q_1}{\bm{u}}\cdot\Delta\bm{u} \;=\; 0 \\
  \label{eq:Q2dlinearized}
  Q_2(\bm{u}_0,d_0+\Delta d;w) &\approx&
  Q_2(\bm{u}_0,d_0;\bm{v}) \;+\; \deriv{Q_2}{d}\Delta d\;=\; 0
\end{eqnarray}
%
from which we obtain the tangent operators
%
\begin{eqnarray}
  \label{eq:Q1dtangent}
  \deriv{Q_1}{\bm{u}}\cdot\Delta\bm{u} &=&
  \int\limits_\Omega \bm{\epsilon}(\Delta\bm{u}) : \left[
    g(d)\dderiv{\Psi_0^+}{\bm{\epsilon}}(\bm{\epsilon}(\bm{u})) +
        \dderiv{\Psi_0^-}{\bm{\epsilon}}(\bm{\epsilon}(\bm{u}))
    \right] : \bm{\epsilon}(\bm{v})\dV \\
  \label{eq:Q2dtangent}
  \deriv{Q_2}{d}\Delta d &=&
  \int\limits_\Omega \left[ \Delta d \left(
    g''(d)\Psi_0^+(\bm{\epsilon}(\bm{u})) + \frac{G_c}{\ell} \right) w \;+\;
    G_c\ell\bm{\nabla}(\Delta d)\cdot\bm{\nabla}w \right] \dV \;+\;
    \frac{1}{\gamma}\int\limits_{{\rm CR}_{l-1}}\!\!\Delta dw \dV
\end{eqnarray}
%
Since $g''(d)=2(1-k)$ is a constant, the latter is linear in $\Delta d$ and $w$,
and independent of $d$, hence a linear solve is sufficient for
Equation~(\ref{eq:Q2dlinearized}).

\bibliographystyle{unsrt}
\bibliography{/home/kmo/LaTeX/bib/article}
\end{document}

This nonlinear equation is solved by Newton-Rapson iterations at each load step,
that is, starting from a known solution $(\bm{u}_i,d_i)$ at iteration $i\ge0$,
we do the linearization
%
\begin{equation}
\label{eq:linearization}
  E'(\bm{u}_i+\Delta\bm{u}_i,d_i+\Delta d_i;\bm{v},w) \;\approx\;
  E'(\bm{u}_i,d_i;\bm{v},w) \;+\;
  \Phi(\Delta\bm{u}_i,\Delta d_i;\bm{u}_i,d_i;\bm{v},w) \;=\; 0
\end{equation}
%
where $\Phi$ is a functional which is linear in the unknowns
$(\Delta\bm{u}_i,\Delta d_i)$ and the test functions $(\bm{v},w)$.

A scalar function $f(\alpha)$ is then introduced for the line search
at iteration $i$
%
\begin{equation}
\label{eq:f}
  f(\alpha) := E(\bm{u}_i+\alpha\Delta\bm{u}_i,d+\alpha\Delta d_i)
\end{equation}
%
such that
%
\begin{equation}
\label{eq:fprime}
  f'(\alpha) := E'(\bm{u}_i+\alpha\Delta\bm{u}_i,d+\alpha\Delta d_i;
                   \Delta\bm{u}_i,\Delta d_i)
\end{equation}
%
The essense of the line search is then to evaluate the Equations~(\ref{eq:f})
and~(\ref{eq:fprime}) at a given number (say 10) of equally distributed points
$\alpha_k\in[-1,1]$ through which a cubic polynomial function $\tilde f(\alpha)$
is fitted. Then we calculate the minimum value
$\alpha^*={\rm argmin}_{\alpha\in[-1,1]} \tilde f(\alpha)$
and the solution at iteration $i+1$ as
$\bm{u}_{i+1} = \bm{u}_i+\alpha^*\Delta\bm{u}_i$ and
$d_{i+1} := d_i+\alpha^*\Delta d_i$.

In the {\sl IFEM} implementation, we use $c=1-d$ as free phase-field variable
instead of $d$, and therefore evaluate $f(\alpha)$ and $f'(\alpha)$ as follows,
with $\bm{u}_\ipa := \bm{u}_i+\alpha\Delta\bm{u}_i$
and $c_\ipa := c_i+\alpha\Delta c_i$:
%
\begin{equation}
\label{eq:fIFEM}
\eqalign{
  f(\alpha) = E(\bm{u}_\ipa,c_\ipa) \;=&\;
  \int\limits_\Omega \left[
    c_\ipa^2\Psi_0^+(\bm{\epsilon}(\bm{u}_\ipa)) +
           \Psi_0^-(\bm{\epsilon}(\bm{u}_\ipa))
  \right]\dV \cr +&\;
  G_c\int\limits_\Omega\left(
  \frac{1}{2\ell}(1-c_\ipa)^2 + \frac{\ell}{2}|\nabla c_\ipa|^2
  \right) \dV + \frac{1}{2\gamma}\int\limits_{{\rm CR}_{l-1}}c_\ipa^2 \dV}
%
\end{equation}
%
\begin{equation}
\label{eq:fprimeIFEM}
\eqalign{
  f'(\alpha) = E'(\bm{u}_\ipa,c_\ipa;\Delta\bm{u}_i,\Delta c_i) \;=&\;
  \int\limits_\Omega \left[
    c_\ipa^2\deriv{\Psi_0^+}{\bm{\epsilon}}(\bm{\epsilon}(\bm{u}_\ipa)) +
    \deriv{\Psi_0^-}{\bm{\epsilon}}(\bm{\epsilon}(\bm{u}_\ipa))
  \right] : \bm{\epsilon}(\Delta\bm{u}_i)\dV \cr +&\;
  \int\limits_\Omega
    2c_\ipa\Psi_0^+(\bm{\epsilon}(\bm{u}_\ipa)) \Delta c_i \dV \cr -&\;
  G_c\int\limits_\Omega\left(
  \frac{1}{\ell}(1-c_\ipa)\Delta c_i - \ell\nabla c_\ipa\cdot\nabla(\Delta c_i)
  \right) \dV \cr +&\;
  \frac{1}{\gamma}\int\limits_{{\rm CR}_{l-1}}c_\ipa\Delta c_i \dV}
\end{equation}
%
Notice the change of sign of the three last terms of
Equation~(\ref{eq:fprimeIFEM}) compared to Equation~(\ref{eq:Eprime}),
which results from $\deriv{c_\ipa}{\alpha}=\Delta c_i=-\Delta d_i$.
The evaluation of Equation~(\ref{eq:fprimeIFEM}) can be carried out as a
dot-product between the internal (residual) force vector and the iterative
solution vector, i.e.
%
\begin{equation}
  f'(\alpha) = {\bf R}_u(\bm{u}_\ipa,c_\ipa)\cdot\Delta{\bf r}^u_i
             + {\bf R}_c(\bm{u}_\ipa,c_\ipa)\cdot\Delta{\bf r}^c_i
\end{equation}
%
where
%
\begin{equation}
  {\bf R}_u(\bm{u},c) := \int\limits_\Omega
     {\bf B}^T\cdot\bm{\sigma}(\bm{\epsilon}(\bm{u}),c) \dV
\end{equation}
%
and
%
\begin{equation}
  {\bf R}_c(\bm{u},c) := \int\limits_\Omega\left[\left(
    2c\Psi^+(\bm{\epsilon}(\bm{u})) - \frac{G_c}{\ell}(1-c)\right){\bf N} +
    G_c\ell\nabla c\cdot\deriv{\bf N}{\bm{x}}
    \right] \dV + \int\limits_{{\rm CR}_{l-1}} \frac{c}{\gamma}{\bf N} \dV
\end{equation}
%
where ${\bf B}$ is the strain-displacement matrix
(i.e., $\bm{\epsilon}(\bm{u})={\bf B}{\bf r}^u$) and
${\bf N}=\{N_1,N_2,\ldots,N_{n_{\rm nod}}\}^T$
denotes the vector of nodal basis functions (i.e., $c={\bf N}^T{\bf r}^c$),
and ${\bf r}^u$ and ${\bf r}^c$ denote vectors of nodal variables for the
respective solution fields $\bm{u}$ and $c$.

In order to solve Equation~(\ref{eq:linearization}) for the unknowns
$\Delta\bm{u}_i$ and $\Delta c_i=1-\Delta d_i$ we first need to develop
the functional $\Phi$.
With $c=1-d$ as the free phase-field variable, we write
%
\begin{equation}
\label{eq:linearizationIFEM}
  E'(\bm{u}_i,c_i;\bm{v},w) \;+\;
  \Phi(\Delta\bm{u}_i,\Delta c_i;\bm{u}_i,c_i;\bm{v},w) \;=\; 0
\end{equation}
%
where
%
\begin{equation}
\label{eq:EIFEM}
\eqalign{
  E(\bm{u},c) \;=&\;
  \int\limits_\Omega \left[
    c^2\Psi_0^+(\bm{\epsilon}(\bm{u})) + \Psi_0^-(\bm{\epsilon}(\bm{u}))
  \right]\dV \cr +&\;
  G_c\int\limits_\Omega\left(
  \frac{1}{2\ell}(1-c)^2 + \frac{\ell}{2}|\nabla c|^2
  \right) \dV + \frac{1}{2\gamma}\int\limits_{{\rm CR}_{l-1}}c^2 \dV}
\end{equation}
%
\begin{equation}
\label{eq:EprimeIFEM}
\eqalign{
  E'(\bm{u},c;\bm{v},w) \;=&\;
  \int\limits_\Omega \left[
    c^2\deriv{\Psi_0^+}{\bm{\epsilon}}(\bm{\epsilon}(\bm{u})) +
    \deriv{\Psi_0^-}{\bm{\epsilon}}(\bm{\epsilon}(\bm{u}))
  \right] : \bm{\epsilon}(\bm{v})\dV \;+\;
  \int\limits_\Omega
    2c\Psi_0^+(\bm{\epsilon}(\bm{u})) w \dV \cr -&\;
  G_c\int\limits_\Omega\left(
  \frac{1}{\ell}(1-c)w - \ell\nabla c\cdot\nabla w
  \right) \dV \;+\;
  \frac{1}{\gamma}\int\limits_{{\rm CR}_{l-1}}cw \dV}
\end{equation}
%
and
%
\begin{equation}
\label{eq:tangent}
  \Phi(\Delta\bm{u},\Delta c;\bm{u},c;\bm{v},w) \;=\;
  \deriv{E'}{\bm{u}}\Delta\bm{u} \;+\; \deriv{E'}{c}\Delta c
\end{equation}
%
Using the chain rule, and that $\deriv{\bm{\epsilon}}{\bm{u}}\cdot\Delta\bm{u}
\equiv\bm{\epsilon}(\Delta\bm{u})$ and $\deriv{\nabla c}{c}\Delta c
\equiv\nabla(\Delta c)$, we obtain
%
\begin{equation}
\label{eq:tangentU}
\eqalign{
  \deriv{E'}{\bm{u}}\Delta\bm{u} \;=&\;
  \int\limits_\Omega \bm{\epsilon}(\Delta\bm{u}) : \left[
  c^2\dderiv{\Psi_0^+}{\bm{\epsilon}}(\bm{\epsilon}(\bm{u})) +
  \dderiv{\Psi_0^-}{\bm{\epsilon}}(\bm{\epsilon}(\bm{u}))
  \right] : \bm{\epsilon}(\bm{v}) \dV \cr +&\;
  \int\limits_\Omega \bm{\epsilon}(\Delta\bm{u}) :
  2c\deriv{\Psi_0^+}{\bm{\epsilon}}(\bm{\epsilon}(\bm{u})) w \dV \cr =&\;
  \Delta{\bf r}^u\left[ \int\limits_\Omega
  {\bf B}^T\deriv{\bm{\sigma}(\bm{\epsilon}(\bm{u}),c)}{\bm{\epsilon}}{\bf B}
  \dV \right] {\bf r}^v \;+\;
  \Delta{\bf r}^u\left[ \int\limits_\Omega
  {\bf B}^T\bm{\sigma}^+(\bm{\epsilon}(\bm{u}),c)
  \dV \right] {\bf r}^w \cr =&\;
  \Delta{\bf r}^u\hskip49pt{\bf K}_{uu}\hskip49pt{\bf r}^v\hskip4pt +\;
  \Delta{\bf r}^u\hskip48pt{\bf K}_{uc}\hskip42pt{\bf r}^w}
\end{equation}
%
and
%
\begin{equation}
\label{eq:tangentC}
\eqalign{
  \deriv{E'}{c}\Delta c \;=&\;
  \int\limits_\Omega
  2c\Delta c\deriv{\Psi_0^+}{\bm{\epsilon}}(\bm{\epsilon}(\bm{u})) :
  \bm{\epsilon}(\bm{v})\dV \;+\;
  \int\limits_\Omega
    2\Delta c\Psi_0^+(\bm{\epsilon}(\bm{u})) w \dV \cr +&\;
  G_c\int\limits_\Omega\left(
  \frac{1}{\ell}\Delta cw + \ell\nabla(\Delta c)\cdot\nabla w
  \right) \dV \;+\;
  \frac{1}{\gamma}\int\limits_{{\rm CR}_{l-1}}\Delta cw \dV \cr =&\;
  \Delta{\bf r}^c\left[ \int\limits_\Omega
  \bm{\sigma}^+(\bm{\epsilon}(\bm{u}),c){\bf B}
  \dV \right] {\bf r}^v \;+\; \Delta{\bf r}^c
  \left[
    \begin{array}{c}\ldots\\\mbox{(fill in content here)}\\\ldots\end{array}
  \right]
  {\bf r}^w \cr =&\;
  \Delta{\bf r}^c\hskip42pt{\bf K}_{uc}^T\hskip42pt{\bf r}^v\hskip4pt +\;
  \Delta{\bf r}^c\hskip50pt{\bf K}_{cc}\hskip45pt{\bf r}^w
}
\end{equation}
%
So then the Equation~(\ref{eq:linearizationIFEM}) is transformed into the
linear system of equations
%
\begin{equation}
  \left\{\begin{array}{c} {\bf R}_u \\ {\bf R}_c \end{array}\right\} \;+\;
  \left[\begin{array}{cc}
  {\bf K}_{uu} & {\bf K}_{uc} \\ {\bf K}_{uc}^T & {\bf K}_{cc}
  \end{array}\right]
  \left\{\begin{array}{c}
  \Delta{\bf r}_u \\ \Delta{\bf r}_c
  \end{array}\right\} \;=\;
  \left\{\begin{array}{c} {\bf 0} \\ {\bf 0} \end{array}\right\}
\end{equation}

\bibliographystyle{unsrt}
\bibliography{/home/kmo/LaTeX/bib/article}
\end{document}
